\sekshun{Restricted Hartree-Fock}
\label{Restricted_Hartree-Fock}
\index{Restricted Hartree-Fock}

ExotiMO solves the time-independent Schr\"{o}dinger equation... Before 
developing the RHF implementation of ExotiMO, it is useful to review... in order to
gain some background and lay some foundation in the notation to be used throughout the
specification.

\section{Background}
\begin{TODO}
Describe exactly what Hartree-Fock method is from Schrodinger to Born-Oppenheimer to 
the Hamiltonian to the determinantal method to variational method to Hartree-Fock. 
Draw from Cook, chapter 1.
\end{TODO}

\begin{TODO}
Derive the general Hartree-Fock equation and the matrix SCF equations. Draw from Cook, 
chapters 2 and 3.
\end{TODO}

\begin{TODO}
Derive the RHF method. Draw from Cook, chapter 4.
\end{TODO}

\begin{TODO}
Describe the units and basis functions to be used (PGBF and CGBFs). Draw from Cook, Appendix to 
chapter 1 and TPMD derivation.
\end{TODO}



\section{ExotiMO Restricted Hartree-Fock Method}
\begin{TODO}
Add considerations (storage, vector/matrix operations, how "object-oriented" the code will be).
Update scope and functional requirements with decisions.
\end{TODO}


