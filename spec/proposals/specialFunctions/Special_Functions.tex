\sekshun{Special Functions}

Special functions are functions for which the compiler provides default definitions.  
Together, these special functions support
value-initialization, default-construction,
copy-construction, destruction, assignment, fieldwise-construction and equality comparisons.
The first four of these are required by the Chapel
compiler.  The remainder are provided as a convenience
to the user.  The compiler-generated version of a special function may always be
overridden by an explicity user-defined version.

Of the special functions required by the compiler, all but assignment are associated with
a state transition in the object lifecycle.  These associations are detailed in the
following subsection.  The remaining sections describe each of the special functions in
turn.  Each section details the interface, semantics and visibility rules for each
function.  For the required special functions, language elements that cause
the compiler to generate a call to that special function are described.


\subsection{The Object Lifecycle}

In general, the states an object can be in are:
\begin{enumerate}
\item Undefined
\item Uninitialized
\item Value-Initialized
\item Field-Initialized
\item Fully-Initialized (a.k.a. Constructed)
\item Destroyed
\item Reclaimed
\end{enumerate}

Before an object has been allocated, it is in the ``undefined'' state.  This is
equivalent to representing its storage with a null pointer; the object {\it per
se} does not exist.

Once memory sufficient to represent the object has been allocated, the object
moves to the ``uninitialized'' state.  The name of the object (or the class
variable) refers to actual memory, but that memory is in an unknown state.

Value-initialization moves the object to the ``value-initialized'' state.  Each
field within an aggregate object is in an initial state consistent with its type,
but neither the fields nor the object as a whole have been ``constructed''.
Value-initialization may be applied knowing only the type of the object --- or
just the types of the fundamental leaves of an aggregate type.  In particular,
neither the constructors nor the field initializers are consulted in creating a
value-initialized object.\footnote{This can be stated more simply using just the
  restriction against consulting constructors, since field-initialization
  depends upon construction.}

Field-initialization consists of bringing each field in an aggregate type into a
state consistent with the field declarations.  If the field is declared without
a specified initializer, then the default constructor (for the type of that
field) is called; otherwise, a constructor for the type of that field is called,
passing the initialization expression for that field as its operand.

Following field-initialization, the actions specified in the body of the
constructor are called.  In many cases, the body of the constructor is empty,
because the actions of value- and field-initialization bring the object into a
state consistent with the desired class invariants.  In any case, after the body
of one of its constructors has run to completeion, the object is
fully-initialized (i.e. constructed).

When an object is destroyed it transitions to the destroyed state.  This state
is equivalent to ``uninitialized'', meaning that its contents cannot be relied
upon.  An implementation may reuse a destroyed object without first returning it
to the heap through deallocation.\footnote{After its destruction, an object may
  still be accessible through the name or reference used to call the
  destructor.  But since its contents cannot be relied upon, it represents a
  programming error to attempt to read or write an object in this state.}

Once an object has been destroyed, an implementation may reuse or reclaim it.
This may happen immediately or at some unspecified future time.  After it has
been reclaimed, the object returns to the undefined state.



\section{Value Initialization}

Value-initialization is provided by the \chpl{_defaultOf()} function.
This function is called wherever the compiler needs to provide value-initialization
for an instance of the corresponding type.  

Value-initialization brings the object into a known initial state before
construction is applied.  For fundamental types (including numerics, strings,
enums and ranges) as well as sync/single variables and atomics, this is the
value specified in section 8.1.1 [Default Initialization].  The remaining type
classes (classes, records, arrays and tuples) will be handled specifically below.
% TODO: Default initialziation as discussed in that section should actually be
% interpreted as default construction.  Default construction is
% value-initialization followed by the actions specified in the constructor
% body.  But since the body of the constructor for every fundamental type is
% empty, the distinction between value-initialization and default-initialization
% is moot.
% The TODO involves separating the concepts of value-initialization and
% default-construction.  Default-initialization is then naturally a combination
% of the two.
For records, arrays and tuples, value-initialization consists of
value-initialization applied element-wise (recursively).  For class variables,
value-initialization sets the variable equal to \chpl{nil}.

Value-initialization is most important in connection with externally defined
types.  It provides the means to bring an object associated with an external
type to a value-initialized state.  Other languages sometimes specify this value
for some or all types; \chpl{_defaultOf} supplies a hook, so the authors of
libraries supporting interoperability can satisfy the external language's
value-initialization expectations.

Although the language currently allows it, overriding the definition
of \chpl{_defaultOf} for fundamental types is strongly discouraged.  For
example, redefining the value returned by
\begin{chapel}
proc _defaultOf(type t) : t where t==bool(?w) ;
\end{chapel}
from \chpl{false} to \chpl{true} would change the initial value of a
Boolean program-wide, which would then break any code that uses the value of any
Boolean variable that is not explicitly initialized.

In terms of the object lifecycle, value-initialization performs the second step.  

\subsection{Interface}

The \chpl{_defaultOf} function has
the following signature.
\begin{chapel}
proc _defaultOf(type t) : t ;
\end{chapel}

\begin{future}

Currently, the interface to \chpl{_defaultOf()} takes a single type argument and
returns an instance that is properly value-initialized.  This means that
creation of an initialized object in general involves either assignment or
copy-construction.  Otherwise, the construction of an object cannot be described
in terms of the language itself.

In the future, it is desired to restate the interface to \chpl{_defaultOf()} as
a member function.  This will permit objects to be initialized in-place, which
will simplify the generated code and improve run-time efficiency.

Rendering the \chpl{_defaultOf()} function in method form will also support
calling \chpl{_defaultOf()} before entering the constructor, rather than calling
it as part of the constructor.  This delineation is important, because the
recursive definitions of \chpl{_defaultOf()} and the compiler-supplied
constructors can then be kept distinct (as opposed to being mutually dependent).

\end{future}

\subsection{Semantics}

Value initializtion is more fundamental than construction.  It is equivalent to
the language concept of default initialization (also known as guaranteed
initialization).  

Value-initialization and construction of an object take place in sequence.  When
an object is created, it starts out as uninitialized storage.  Value
initialization places the object (or each field of an aggregate object,
recursively) into a known initial state.  Implicitly, it is possible to
determine the initial value for each field in object based solely on its type, whereas
construction may establish relationships between fields that require treating
the object as a whole (and may involve external knowledge as well).

\begin{note}
At present, the order in which value-initialization is carried out among the
fields is unspecified.  This is intended to give an implementation considerable
flexibility in implementing value-initialization.

Unless overridden, value-initialization can most-often be found (at compile
time) to be equivalent to zero-initialization.  If this is the case, the
provision for value-initialization can be satisfied by emitting the equivalent
of a \chpl{memset(var, 0, sizeof(var));} or performing the object allocation
using \chpl{calloc()}.
\end{note}

\subsection{Visibility}

The \chpl{_defaultOf()} function must be invoked with an operand that evaluates
to a concrete type $T$.  Generic instantiation rules are applied to all
user-defined definitions of \chpl{_defaultOf()} with compatible
signatures\footnote{That is, functions named \chpl{_defaultOf} that can accept a type
  argument as their first argument and whose other arguments all have default
  values associated with them.} to generate a candidate set.  If the candidate
set is non-empty, normal resolution rules are applied to select the best
candidate.

\subsection{Usage}

To enforce value-initialization within a constructor, a call to \chpl{_defaultOf()} is
inserted by the compiler at the start of every (user-supplied as well as
compiler-generated) constructor.  The \chpl{_defaultOf()} function should not be
invoked directly from user code.

\begin{rationale}

The \chpl{_defaultOf()} function may assume that its operand is not yet
value-initialized.  This condition can only occur within a constructor.  But since
\chpl{_defaultOf()} is automatically invoked at (or preferably, before) the
beginning of every constructor, it would be pointless to invoke it explicitly in
that context. Fields in an aggregate type being constructed have also already been
value-initialized through recursive calls to \chpl{_defaultOf()}.

\end{rationale}



\section{Default Construction}

Default construction for a given type is provided by any constructor compatible
with that type that can be called with an empty argument list (i.e. ``no
arguments'' or ``zero arguments'').  In particular, a generic constructor may be
compatible with many concrete types, and a constructor with one or more formals
may be called with no arguments so long as each argument has an associated
default value.

\subsection{Interface}

An explicit default constructor for a type $T$ has the following signature.
\begin{chapel}
proc T.T();
\end{chapel}
When the compiler supplies a default constructor, it will have exactly this
signature.  However, as explained above, any user-defined signature that can be
called with an empty argument list will serve as a default constructor.

\subsection{Semantics}

The intended semantics of the default constructor are to take an object from the
value-initialized state to a fully-constructed state consistent with the
corresponding type.  For a user-defined type, this typically means adjusting the
initialization of the object to satisfy the class (or record) invariants.

Construction proceeds in two phases: field-initialization and
instance-initialization.  Field-initialization moves the object from the
value-initialized state to the field-initialized state.  This is done by
recursively applying either the default constructor or the copy constructor to
each contained field, depending on whether the field is declared without an
initializer or with an initializer, respectively.  Instance-initialization
creates further relationships between the fields in an object, to establish the
invariants that are common to all fully-initialized objects in that class.
Details on the syntax and semantics of user-defined constructors are provided in
Section 15.3 [Class Constructors].

To remain consistent with current behavior, field-initialization is performed
sequentially, in the order that the variables are declared in the class or
record definition.  This naturally satisfies the author's expectations, by
analogy with the order of initialization of successive declarations 
in a procedure definition.

When present, the compiler-generated constructor for a class or record type
consists of field-initialization followed by an empty constructor body.  That
is, instance-initialization is trivial.
\footnote{At present, there is no way to specify that the compiler should not generate a
default constructor if no user-callable version is provided.  Thus, in order to
prevent the creation of a default-valued object, the user must currently define
a default constructor that generates a compile-time error.}
This choice is reasonable for the
compiler's default behavior, since the compiler cannot infer any relationships
among the fields based solely on their types.  If such relationships
(invariants) are to be established, they must be provided explicitly by the
class author.

\begin{note}

Following field-initialization, each field in an aggregate object is considered
to be initialized.  Therefore, updates to these fields in the body of a
constructor are performed using assignment.

\end{note}

\subsection{Visibility}

If no user-defined constructors are supplied for a given type, then the compiler
provides an all-fields constructor as detailed below.  Because the all-fields
constructor may be invoked with an empty argument list, it implements the
default constructor in this case.

If the user-defined type defines at least one constructor, then the compiler
will provide a default constructor (as defined above).  During resolution of a
constructor called with no arguments, the user-defined constructors for that
concrete type will be considered first (including ones contributed from
compatible generic types).  If no matching constructor is found in that set,
then the compiler-generated default constructor is considered.

\subsection{Usage}

The compiler implicitly invokes the default constructor when a variable is
declared without an explicit initializer.  Also, it is used in the recursive
definition of the default constructor for an aggregate type.



\section{Copy Construction}

Copy construction for a given type is provided by any constructor compatible
with that type that can be called with an argument list consisting of a single
object of that type.  The selected definition may involve generic types for both
the object and the operand.  The definition may also involve other arguments, so
long as all but one of the arguments has a default value associated with it, and
the type of the remaining formal argument can bind to an actual argumet of the
object type.

\subsection{Interface}

The copy constructor for a type $T$ has the following signature
\begin{chapel}
proc T.T(const ref T);
\end{chapel}
When the compiler supplies a copy constructor, it will have exactly this
signature.  However, as explained above, any user-defined signature that can be
called with an argument list containing just one argument coercible to the
object type will serve.

\subsection{Semantics}

The intended semantics of the copy constructor are to create a fully-constructed
object that is a verbatim copy of the operand.  The compiler uses copy
construction to implement pass-by-value and return-by-value semantics.  This
corresponds to formal arguments whose concrete intent resolves to \chpl{in}, and
to records and other value types when they are returned by value.\footnote{The
return intent \chpl{ref} causes the return value to be returned by reference.}

As with default construction, copy construction proceeds in two phases:
field-initialization followed by instance-initialization.  Field initialization
is exactly the same as for default construction described above.  Instance
initialization then provides the copy semantics and provides the invariants
required by the class design.  As with the body of the default constructor
updates to fields within the object are treated as assignments (initialization
already having been performed during field-initialization).

When present, the compiler-generated copy constructor for a class or record type
performs a field-by-field copy of the fields in the operand into the
corresponding fields in the object being initialized.  

\begin{note}
It may be observed that the compiler-generated copy constructor overwrites every
field, even though they have already been initialized during
field-initialization.  So long as the semantics remain unchanged, the compiler
may skip field-initialization and use copy-initialization instead to establish
the value of each field.

This optimization cannot be applied in the general case, where the behavior of
default-initialization, copy-initialization or assignment has been overridden by
the user for the type of a given field.  But it may safely be applied where
assignment and the default constructor (for default-initialization) or the copy
constructor (for copy-initialization) have been supplied by the compiler for the
corresponding field type.
\end{note}

\subsection{Visibility}

The compiler provides copy constructor for every type, defined as specified
above.  During resolution of a constructor call, the compiler-supplied copy
constructor is considered only if there are no candidates among the user-defined
constructors applicable to that object type.

\subsection{Usage}

The compiler implicitly invokes the copy constructor to initialize a formal
argument with in intent from the corresponding actual argument.  It is also used
to initialize the returned value temporary variable (in the calling context)
from the return value expression in a called routine.



\section{All-Fields Construction}

The all-fields constructor is provided by the compiler as a convenience.  It
supports rapid prototyping by allowing client code to specify the initial states
of fields within an object being constructed, without requiring the class author
to supply this capability explicitly.

The all-fields constructor is provided only if no user-specified constructors
have been defined for that type.  

\begin{rationale}

It is a useful prototyping feature to be able to structure data (as in a C
\chpl{struct}) without having to write a constructor and accessor functions to
support its initialization and use.

On the other hand, allowing client code direct access to the fields in a class
or record defeats encapsulation.  The compromise is to provide the all-fields
constructor when no user-defined constructors are present, but to hide this
after any user-defined constructor is provided.  The operative assumption is
that a user who is adding constructors to a class will be aware of the
all-fields constructor and the conditions under which it is provided.

\end{rationale}

\subsection{Interface}

For a class or record type, the signature of the all-fields constructor contains
one formal argument for each field in the aggregate, arranged in declaration
order.  The name of each formal is exactly the same as the name of the
corresponding field.  Each formal is provided with a default value in the same
manner as for field-initialization.  That is, default construction provides the
initial value for formals whose fields are declared without an initializer, and
copy construction provides the initial value for formals whose fields are
declared with an initializer. 

\subsection{Semantics}

All-fields construction proceeds in two phases: field-initialization followed by
instance-initialization.  Field-initialization is provided as described above.
Instance initialization consists of overwriting each field with the value of the
corresponding formal.

\begin{note}

As with the copy constructor, it can be observed that field-initialization is
essentially moot, because the semantics of the all-fields constructor cause each
field to be overwritten in the body of the constructor.

Similarly, if assignment and copy construction or default construction
(corresponding to whether the associated field has an initializer or not) are
not overridden by the user, the difference between performing field
initialization or not for that field will not be observable by the user.  Under
those conditions, the initialization of that field may be elided.

\end{note}

\subsection{Visibility}

The all-fields constructor is visible only if no user-defined constructors have
been provided for the corresponding class or record type.

\subsection{Usage}

The compiler does not generate any calls to the all-fields constructor.



\section{Assignment}

Assignment is provided by the procedure named ``='' whose left and right
operands are compatible with the object type.  Assignment is used to replace the
value of a field or variable that has already been initialized.  It is used by
the compiler to update the value of a field in the body of a copy constructor or
the all-fields constructor.  It is also used to update the value of the return
value temporary prior to its return to the caller.

\begin{future}

In addition to the two-argument form, it is desirable to support assignment as a
method.  Given that the canonical form for assignment has been updated to pass
the left operand by reference, the two forms are equivalent in behavior and
efficiency.

The return value variable is useful in determining the type of the return value
of a routine where that is unspecified.  However, by using the semantics that a
return statement directly calls a copy-constructor to transfer the result to the
calling context would eliminate an unnecessary assignment from the semantics of
that statement.

\end{future}

\subsection{Interface}

The canonical two-argument form for assignment is:
\begin{chapel}
proc =(ref lhs, rhs) : void ;
\end{chapel}

\subsection{Semantics}

Excluding class variables, the semantics of assignment are to replace the
contents of the lhs operand with the contents of the rhs operand.  For class
variables, the class instance reference held by the lhs operand is replaced by a
copy of the reference held by the rhs operand.  The object referenced by the rhs
thus becomes shared through the lhs, and the object formerly accessible through
the lhs operand is no longer accessible through that name (though it may be
accessible through other paths).

For fundamental types (excluding strings),
assignment can be accomplished by a bitwise copy.  
Depending on how strings are implemented, a bitwise copy may or may not suffice.

When provided, the compiler-supplied version of record assignment consists of a
(recursive) field-wise assignment from the fields in the right operand to the
fields in the left operand.

\begin{note}

As with the copy and all-fields constructors, it can be observed that a bitwise
copy will suffice as long as none of the fields (recursively) have types that
have user-defined assignment operators associated with them.  The definition of
the compiler-generated record assignment operator can be seen to be equivalent
ot a bitwise copy.

\end{note}

\subsection{Visibility}

A compiler-generated assignment operator is provided for every type except class
types.  However, a compatible user-defined assignment operator will always
override the compiler-generated one.

\subsection{Usage}

The compiler currently generates assignments to implement the bodies of the copy
constructor and all-fields constructor.  Also, as mentioned above, it is used
to establish the value of the return value temporary, prior to its return to the
caller.



\section{Destruction}

Destruction is provided by the destructor.  Its purpose is to move a
fully-constructed object into the destroyed state.  Before transitioning to this
state, an object should release all resources dynamically attached to it during
the course of execution.

Once an object is destroyed, it is available for reclamation, so its contents
should not be relied upon.  It is a programming error to read or write an object
that has already been destroyed.

\subsection{Interface}

The destructor for a type \chpl{T} is a method that takes no arguments and has
the name \chpl{\~T}.

\begin{chapel}
proc T.~T();
\end{chapel}

\begin{openissue}
Polymorphic behavior for destructors is not yet specified, but is essential for
supporting inhomogeneous containers.  It may be sufficient to specify that
destructors are always dynamically dispatched, though this carries some
unnecessary overhead when the destructor can be statically bound.
\end{openissue}

\subsection{Semantics}

The semantics of the destructor are to release any resources owned by the object
before the object itself is release.  Once released, the object may (though need
not necessarily) be reclaimed.  Reclamation involves placing the memory formerly
allocated to the object back on the heap (for dynamically allocated objects).
Reclamation and reuse of the memory happens automatically when the object is
allocated on the stack.

Since the compiler cannot know about additional resources allocated to the
object during execution, the compiler-generated destructor is trivial.

\subsection{Visibility}

The compiler-generated destructor is supplied for every type.  However during
resolution, if a compatible user-defined destructor is available, it is always
selected in preference to the compiler-generated one.

\subsection{Usage}

The compiler calls the destructor for non-class objects when execution exits the
scope in which the name bound to each such object is defined.  Variables defined
at global (module) scope are destroyed before the program terminates.  For class
objects, the destructor is invoked explicitly using the \chpl{delete} operator.



% TODO: This section is at least partially redundant with the section 15.4.
% Probably, the scope of this appendix should be reduced to UMM-related functions and any
% new material here merged into section 15.4..
\section{Field Accessors}

For each field in an aggregate type, the compiler provides a field accessor
function.  For read-only fields, the accessor returns the value of that field.

For read/write fields, the accessor also possesses a hidden param bool argument
named \chpl{setter}.  When the corresponding hidden actual is \chpl{true}, the function
returns a reference to the field that can be updated.  When the corresponding hidden
actual is \chpl{false}, the function returns the value of that field.

Read-only fields are fields that are declared using the \chpl{param} or \chpl{type}
keyword.  The implicit base class member \chpl{super} in a derived class or record type is
also a read-only field.

\subsection{Interface}

All field accessor functions are parentheses-less functions, meaning that they must be
called without an argument list (not even an empty one).  The signature of a field
accessor for a read-only field is:
\begin{chapel}
proc ro_field : t ;
\end{chapel}
\noindent
The signature of a field accessor function for a read/write field is equivalent to:
\begin{chapel}
pragma "no parens"
proc rw_field(param setter : bool) : ref t ;
\end{chapel}
\noindent
In both signatures, the return type \chpl{t} is the same as the type of the named field.

When calling the accessor for a read-only field, or the accessor for a read-write field in
an rvalue context, it is called using the normal parentheses-less form.  In that context,
the reference result is implicitly dereferenced to yield a value result.  For example, the
code
\begin{chapel}
record R { var a = 3 ; }
writeln(r.a);
\end{chapel}
\noindent
will print out a \chpl{3}.

When calling the accessor for a read-write field in an lvalue context, the compiler
supplies an implicit \chpl{setter} argument with the value \chpl{true}.  For example,
\begin{chapel}
r.a = 7;
\end{chapel}
\noindent
is implicitly converted to
\begin{chapel}
r.a(setter=true) = 7;
\end{chapel}

\subsection{Semantics}

A getter returns the field of the same name.  If the field is a param, it is
returned as a param.  If the field is a type field, it is returned as a type.
If the field is the superclass object, it is returned by value (i.e. the object
itself, rather than a reference to that object).  Otherwise, the field is
returned by reference.  In this last case, if the reference is used in an rvalue context,
it is implicitly dereferenced to yield a value (i.e. an rvalue).

\subsection{Visibility}

For each field in an aggregate type (class, record or union), the compiler supplies a
default version of the field accessor function.  The semantics of this function are to
always return the value of the named field (for read-only fields) or a reference to the
named field (for read-write fields).  In the default version for read-write fields,
the \chpl{setter} param argument is ignored.

The compiler-supplied version can be overridden by a user-defined version.  In the case of
a read-write field, the \chpl{setter} argument can be used to perform different actions
depending on whether the field is to be written or read.

\subsection{Usage}

The compiler implicitly calls the getter when a field in an aggregate is read,
and implicitly calls the setter when a field is written.  That is, when used in
an rvalue context, the expression \chpl{c.a} is translated internally into
a call to
\begin{chapel}
proc c.a : ref a.type where setter==false;
\end{chapel}
and when used in an lvalue context is translated internally to a call to
\begin{chapel}
proc c.a : ref a.type where setter==true;
\end{chapel}



\section{Casts}

Casts alter the type of an object, and where necessary change its representation as well.
They are called implicitly by the compiler to implement coercions.

\subsection{Interface}

A cast is a procedure that takes a type argument and a value argument.  The return value
is the input value cast to the specified type.
\begin{chapel}
proc _cast(type t, x) : t ;
\end{chapel}

\subsection{Semantics}

The \chpl{_cast} function performs whatever transformations are necessary to convert the
input value \chpl{x} to the specified target type.  The return value is of the target
type.  Casts may involve a loss of information (down-cast) or the insertion of default
information (up-cast).  In either case, since the result needs to be a valid object of
\chpl{type t}, the actions performed by an up-cast are closely associated with the target
\chpl{type t}.  If a down-cast is the reverse of an up-cast, it may also be most
appropriately associated with the definition of \chpl{type t}.

\subsection{Visibility}

The compiler provides default implementations of the \chpl{_cast} function for:
\begin{itemize}
\item Converting integral types to enumerated types.
\item Converting a string to an enumerated type.
\item Converting a enumerator into a string.
\item Converting a record to one of its subtypes.
\end{itemize}

The functions for casting from an integer or string into an enumerated type may be
overridden by a user-defined version.  At present, the record cast function cannot be
overridden, nor can the function for converting an enumerator to a string.

\subsection{Usage}

The compiler calls the integral type to enumerated cast when an integer is bound to a
formal argument with an enumeration type (including the RHS argument of an assignment).
The compiler call the string to enumerated cast when converting a string (e.g. from input)
to an enumerated type.  
The compiler calls the enumerator to string cast when converting
an enumerator to a string (either internally or in formatted I/O).  
The compiler calls the record cast function when a record is coerced to one of its
subtypes.



\section{Hash Functions}

Record hash functions provide a default hashing function for records.  It is used in the
implementation of the DefaultAssociative domain to create a hash address from a key that
is a record.

\subsection{Interface}

The default hash function has the signature:
\begin{chapel}
proc chpl__defaultHash(x) : int(64) ;
\end{chapel}

\subsection{Semantics}

The default hash function produces a 64-bit hash value based on the contents of the given
record.  The hash value is not guaranteed to be unique.  That is, once a hash bucket is
located, elements in that bucket must still be compared for an exact match with the key.

\subsection{Visibility}

The compiler-generated default hash function cannot currently be overridden.

\subsection{Usage}

This function is provided as a convenience to the implementation of the DefaultAssociative
domain type.  It is used for generating hash addresses from keys to support hashed
insertion and lookup.
% Note: Since a field iterator is now available, the compiler-generated version of this
% could probably be replaced by a generic version in module code.


\section{Record Equality}

The compiler provides record equality \chpl{==} and inequalty \chpl{!=} functions as a
convenience.

\subsection{Interface}

The record equality and inequality functions have the signatures:
\begin{chapel}
proc ==(a : R, b : R) : bool ;
proc !=(a : R, b : R) : bool ;
\end{chapel}

\subsection{Semantics}

The record equality function performs a recursive field-wise comparison between the
operands and return \chpl{true} if they all compare equal and \chpl{false} otherwise.  The
output of the record inequality function is symmetrical, returning \chpl{false} if all of
the fields compare equal and \chpl{true} otherwise.

\subsection{Visibility}

Both record equality and record inequality functions may be overridden.

\subsection{Usage}

The compiler and module code may call record equality and inequality functions from
several locations.


