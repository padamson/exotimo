\sekshun{Locales}
\label{Locales_Chapter}

Chapel provides high-level abstractions that allow programmers to
exploit locality by controlling the affinity of both data and tasks to
abstract units of processing and storage capabilities
called \emph{locales}.  The \emph{on-statement} allows for the
migration of tasks to \emph{remote} locales.

\index{local}
\index{remote}
\index{locales!local}
\index{locales!remote}
Throughout this section, the term \emph{local} will be used to
describe the locale on which a task is running, the data located on
this locale, and any tasks running on this locale.  The
term \emph{remote} will be used to describe another locale, the data
on another locale, and the tasks running on another locale.

\section{Locales}
\label{Locales}
\index{locales}

A \emph{locale} is a portion of the target parallel architecture that
has processing and storage capabilities.  Chapel implementations
should typically define locales for a target architecture such that
tasks running within a locale have roughly uniform access to values
stored in the locale's local memory and longer latencies for accessing
the memories of other locales.  As an example, a cluster of multicore
nodes or SMPs would typically define each node to be a locale.  In
contrast a pure shared memory machine would be defined as a single
locale.

\subsection{Locale Types}
\label{The_Locale_Type}
\index{locale@\chpl{locale}}
\index{types!locale@\chpl{locale}}

The identifier \chpl{locale} is a class type that abstracts a
locale as described above.  Both data and tasks can be associated with
a value of locale type.  A Chapel implementation may define subclass(es)
of \chpl{locale} for a richer description of the target architecture.

\subsection{Locale Methods}
\label{Locale_Methods}
\index{locales!methods}
\index{predefined functions!locale}

The locale type supports the following methods:

%% \begin{protohead}
%% proc locale.blockedTasks() : int ;
%% \end{protohead}
%% \begin{protobody}
%% Returns the number of tasks on this locale which are blocked at the time of the
%% call.
%% \end{protobody}

\index{predefined functions!callStackSize@\chpl{callStackSize}}
\index{locales!callStackSize@\chpl{callStackSize}}
\begin{protohead}
proc locale.callStackSize: uint(64);
\end{protohead}
\begin{protobody}
Returns the per-task call stack size used when creating tasks on the
locale in question.  A value of 0 indicates that the call stack size
is determined by the system.
\end{protobody}

\index{predefined functions!id@\chpl{id}}
\index{locales!id@\chpl{id}}
\begin{protohead}
proc locale.id: int;
\end{protohead}
\begin{protobody}
Returns a unique integer for each locale, from 0 to the number of
locales less one.
\end{protobody}

%% \begin{protohead}
%% proc locale.idleThreads() : uint ;
%% \end{protohead}
%% \begin{protobody}
%% Returns the number of threads which are currently idle on this locale.
%% \end{protobody}

\index{predefined functions!maxTaskPar@\chpl{maxTaskPar}}
\index{locales!maxTaskPar@\chpl{maxTaskPar}}
\begin{protohead}
proc locale.maxTaskPar: int(32);
\end{protohead}
\begin{protobody}
Returns an estimate of the maximum parallelism available for tasks
on a given locale.
\end{protobody}

\index{predefined functions!name@\chpl{name}}
\index{locales!name@\chpl{name}}
\begin{protohead}
proc locale.name: string;
\end{protohead}
\begin{protobody}
Returns the name of the locale.
\end{protobody}

\index{predefined functions!numCores@\chpl{numCores}}
\index{locales!numCores@\chpl{numCores}}
\begin{protohead}
proc locale.numCores: int;
\end{protohead}
\begin{protobody}
Returns the number of logical CPUs available on a given locale.
\end{protobody}

\index{predefined functions!physicalMemory@\chpl{physicalMemory}}
\index{locales!physicalMemory@\chpl{physicalMemory}}
\begin{protohead}
use Memory;
proc locale.physicalMemory(unit: MemUnits=MemUnits.Bytes, type retType=int(64)): retType;
\end{protohead}
\begin{protobody}
Returns the amount of physical memory available on a given locale in
terms of the specified memory units (Bytes, KB, MB, or GB) using a
value of the specified return type.
\end{protobody}

%% \begin{protohead}
%% proc locale.queuedTasks() : uint ;
%% \end{protohead}
%% \begin{protobody}
%% Returns the number of tasks on this locale which are currently on the task queue.
%% \end{protobody}

%% \begin{protohead}
%% proc locale.runningTasks() : uint ;
%% \end{protohead}
%% \begin{protobody}
%% Returns the number of tasks on this locale that are currently running.
%% \end{protobody}

%% \begin{protohead}
%% proc locale.totalThreads() : uint ;
%% \end{protohead}
%% \begin{protobody}
%% Returns the total number of threads (active + idle) that currently exist on this
%% locale.
%% \end{protobody}

\subsection{The Predefined Locales Array}
\label{Predefined_Locales_Array}
\index{Locales@\chpl{Locales}}
\index{numLocales@\chpl{numLocales}}
\index{execution environment}

Chapel provides a predefined environment that stores information about
the locales used during program execution.  This {\em execution
environment} contains definitions for the array of locales on which
the program is executing (\chpl{Locales}), a domain for that array
(\chpl{LocaleSpace}), and the number of locales (\chpl{numLocales}).
\begin{chapel}
config const numLocales: int;
const LocaleSpace: domain(1) = [0..numLocales-1];
const Locales: [LocaleSpace] locale;
\end{chapel}
When a Chapel program starts, a single task executes \chpl{main}
on \chpl{Locales(0)}.

Note that the Locales array is typically defined such that distinct
elements refer to distinct resources on the target parallel
architecture.  In particular, the Locales array itself should not be
used in an oversubscribed manner in which a single processor resource
is represented by multiple locale values (except during development).
Oversubscription should instead be handled by creating an aggregate of
locale values and referring to it in place of the Locales array.

\begin{rationale}
This design choice encourages clarity in the program's source text and
enables more opportunities for optimization.

For development purposes, oversubscription is still very useful and
this should be supported by Chapel implementations to allow
development on smaller machines.
\end{rationale}

\begin{example}
The code
\begin{chapel}
const MyLocales: [0..numLocales*4] locale 
               = [loc in 0..numLocales*4] Locales(loc%numLocales);
on MyLocales[i] ...
\end{chapel}
defines a new array \chpl{MyLocales} that is four times the size of
the \chpl{Locales} array.  Each locale is added to
the \chpl{MyLocales} array four times in a round-robin fashion.
\end{example}

\subsection{The {\em here} Locale}
\label{here}
\index{here@\chpl{here}}
\index{locales!here@\chpl{here}}

A predefined constant locale \chpl{here} can be used anywhere in a
Chapel program.  It refers to the locale that the current task is
running on.

\begin{example}
The code
\begin{chapel}
on Locales(1) {
  writeln(here.id);
}
\end{chapel}
results in the output \chpl{1} because the \chpl{writeln} statement is
executed on locale 1.
\end{example}

The identifier \chpl{here} is not a keyword and can be overridden.

\subsection{Querying the Locale of an Expression}
\label{Querying_the_Locale_of_a_Variable}
\index{locale@\chpl{locale}}

The locale associated with an expression (where the expression is
stored) is queried using the following syntax:
\begin{syntax}
locale-access-expression:
  expression . `locale'
\end{syntax}
When the expression is a class, the access returns the locale on which
the class object exists rather than the reference to the class.  If
the expression is a value, it is considered local.  The implementation
may warn about this behavior.  If the expression is a locale, it is
returned directly.

\begin{example}
Given a class C and a record R, the code
\begin{chapel}
on Locales(1) {
  var x: int;
  var c: C;
  var r: R;
  on Locales(2) {
    on Locales(3) {
      c = new C();
      r = new R();
    }
    writeln(x.locale.id);
    writeln(c.locale.id);
    writeln(r.locale.id);
  }
}
\end{chapel}
results in the output
\begin{chapelprintoutput}{}
1
3
1
\end{chapelprintoutput}
The variable \chpl{x} is declared and exists on \chpl{Locales(1)}.
The variable \chpl{c} is a class reference.  The reference exists
on \chpl{Locales(1)} but the object itself exists
on \chpl{Locales(3)}.  The locale access returns the locale where the
object exists.  Lastly, the variable \chpl{r} is a record and has
value semantics.  It exists on \chpl{Locales(1)} even though it is
assigned a value on a remote locale.
\end{example}

Global (non-distributed) constants are replicated across all locales.
\begin{example}
For example, the following code:
%
% We can't yet specify multiple .good files or .numlocales files, so
% add this test later when we can.
%
\begin{chapel}
const c = 10;
for loc in Locales do on loc do
    writeln(c.locale.id);
\end{chapel}
outputs
\begin{chapelprintoutput}{}
0
1
2
3
4
\end{chapelprintoutput}
when running on 5 locales.
\end{example}


\section{The On Statement}
\label{On}
\index{on@\chpl{on}}
\index{statements!on@\chpl{on}}

The on statement controls on which locale a block of code should be
executed or data should be placed.  The syntax of the on statement is
given by
\begin{syntax}
on-statement:
  `on' expression `do' statement
  `on' expression block-statement
\end{syntax}
The locale of the expression is automatically queried as described
in~\rsec{Querying_the_Locale_of_a_Variable}.  Execution of the
statement occurs on this specified locale and then continues after
the \chpl{on-statement}.

Return statements may not be lexically enclosed in on statements.
Yield statements may only be lexically enclosed in on statements in
parallel iterators~\rsec{Parallel_Iterators}.

\subsection{Remote Variable Declarations}
\label{remote_variable_declarations}
\index{variable declarations!remote}

By default, when new variables and data objects are created, they are
created in the locale where the task is running.  Variables can be
defined within an \sntx{on-statement} to define them on a particular
locale such that the scope of the variables is outside
the \sntx{on-statement}.  This is accomplished using a similar syntax
but omitting the \chpl{do} keyword and braces.  The syntax is given
by:
\begin{syntax}
remote-variable-declaration-statement:
  `on' expression variable-declaration-statement
\end{syntax}
