\sekshun{Organization}
\label{Organization}
\index{organization}

This specification is organized as follows:

\begin{description}
\item[Part I] Introduction

\begin{description}
\item[Chapter~\ref{Purpose}] Purpose, describes the purpose of this document.

\item[Chapter~\ref{Acknowledgments}] Acknowledgements, offers a note of
thanks to people and projects.

\item[Chapter~\ref{Notation}] Notation, introduces the notation that is used
throughout this document.

\item[Chapter~\ref{Organization}] Organization, describes the contents of
each of the parts and chapters within this document.

\item[Chapter~\ref{Development_Approach}] Development Approach, describes the hybrid
Test-Driven Development and Literate Programming approach.
\end{description}

\item[Part II] Requirements Specification

\begin{description}
\item[Chapter~\ref{Scope}] Scope, describes the scope of the ExotiMO package.
\end{description}

\item[Part III] Technical Specification
\begin{description}
\item[Chapter~\ref{Restricted_Hartree-Fock}] Restricted Hartree-Fock, development of the 
Restricted Hartree-Fock driver.

\item[Chapter~\ref{Auxiliary_Functions}] Auxiliary Functions, development of auxiliary functions.
\end{description}

\item[Appendix~\ref{Syntax}] Collected Lexical and Syntax Productions,
contains the syntax productions listed throughout this specification
in both alphabetical and depth-first order.

\end{description}
