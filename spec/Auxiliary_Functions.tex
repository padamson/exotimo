%This is a comment
\sekshun{Auxiliary Functions}
\label{Auxiliary_Functions}
\index{auxiliary functions}
\index{functions!auxiliary}

ExotiMO relies on several auxiliary functions. This chapter details those functions.

\section{Factorial}
\index{auxiliary functions!factorial}
\index{factorial}

\begin{chapeltest}{factorialTest.chpl}
In the code
\begin{chapelpre}
\end{chapelpre}
\begin{chapel}
use auxiliary;
for n in (2, 4, 9) do
  writeln(factorial(n));
\end{chapel}
we use the factorial function to write $2!$, $4!$, and $9!$ out on new lines.
\begin{chapelpost}
\end{chapelpost}
\begin{chapeloutput}
2
24
362880
\end{chapeloutput}
\end{chapeltest}

The code that provides the $n!$ function is straightforward:
\begin{chapelsource}{auxiliary.chpl}
\begin{chapel}
proc factorial(n: int): int{
  if (n <= 1) then return 1;
  return n*factorial(n-1);
}
\end{chapel}
\end{chapelsource}

\section{Binomial}
\index{auxiliary functions!binomial function}
\index{binomial function}

We need a function, \lstinline{binomial}, that computes the binomial coefficient $\binom{m}{n}$.
\begin{chapeltest}{binomialTest.chpl}
In the code
\begin{chapelpre}
\end{chapelpre}
\begin{chapel}
use auxiliary;
for m in (10, 15, 20) do {
  for n in (3, 5, 7) do {
    writeln(binomial(m,n));
  }
}
\end{chapel}
we use the \lstinline{binomial} function to output $\binom{m}{n}$ 
for $m\in\{10, 15, 20\}$ and $n \in \{3, 5, 7\}$.
\begin{chapelpost}
\end{chapelpost}
\begin{chapeloutput}
120
252
120
455
3003
6435
1140
15504
77520
\end{chapeloutput}
\end{chapeltest}

The code that provides the \lstinline{binomial} function is straightforward 
and requires the \lstinline{factorial} function:
\begin{chapelsource}{auxiliary.chpl}
\begin{chapel}
proc binomial(m: int, n: int): int {
  return factorial(m)/(factorial(n)*factorial(m-n));
}
\end{chapel}
\end{chapelsource}


\section{Double Factorial}
\index{auxiliary functions!double factorial}
\index{double factorial}

\begin{chapeltest}{doubleFactorialTest.chpl}
In the code
\begin{chapelpre}
\end{chapelpre}
\begin{chapel}
use auxiliary;
for n in (2, 4, 9) do
  writeln(doubleFactorial(n));
\end{chapel}
we use the doubleFactorial function to write $2!!$, $4!!$, and $9!!$ out on new lines.
\begin{chapelpost}
\end{chapelpost}
\begin{chapeloutput}
2
8
945
\end{chapeloutput}
\end{chapeltest}

The code that provides the $n!!$ function:
\begin{chapelsource}{auxiliary.chpl}
\begin{chapel}
proc doubleFactorial(n: int): int{ 
  if (n <= 1) then return 1;
  return n*doubleFactorial(n-2);
}
\end{chapel}
\end{chapelsource}
