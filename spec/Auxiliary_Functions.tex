\sekshun{Auxiliary Functions}
\label{Auxiliary_Functions}
\index{auxiliary functions}
\index{functions!auxiliary}

ExotiMO relies on several auxiliary functions. This chapter details those functions.

\section{Factorial}
\index{auxiliary functions!factorial}
\index{factorial}

\begin{chapelexample}{factorialTest.chpl}
In the code
\begin{chapelpre}
use auxiliary;
\end{chapelpre}
\begin{chapel}
var a: int = factorial(2);
var b: int = factorial(4);
\end{chapel}
we want the factorial function to calculate $2!$ and $4!$, storing the results in \lstinline{a}
and \lstinline{b}, respectively.
\begin{chapelpost}
writeln(a);
writeln(b);
\end{chapelpost}
\begin{chapeloutput}
2
24
\end{chapeloutput}
\end{chapelexample}

The code that provides the $n!$ function is straightforward:
\begin{chapelsource}{auxiliary.chpl}
\begin{chapel}
proc factorial(n: int): int{
  if (n <= 1) then return 1;
  return n*factorial(n-1);
}
\end{chapel}
\end{chapelsource}


\section{Binomial}

We need a function, \lstinline{binomial}, that computes the binomial coefficient $\binom{m}{n}$.
\begin{chapelexample}{binomialTest.chpl}
In the code
\begin{chapelpre}
use auxiliary;
\end{chapelpre}
\begin{chapel}
var a: int = binomial(100,2);
var b: int = binomial(50,3);
\end{chapel}
we want the binomial function to calculate $\binom{100}{2}$ and $\binom{50}{3}$, storing the results in \lstinline{a}
and \lstinline{b}, respectively.
\begin{chapelpost}
writeln(a);
writeln(b);
\end{chapelpost}
\begin{chapeloutput}
4950
19600
\end{chapeloutput}
\end{chapelexample}

The code that provides the binomial function is straightforward and requires the factorial function:
\begin{chapelsource}{auxiliary.chpl}
\begin{chapel}
proc binomial(m: int, n: int): int {
  return factorial(m)/(factorial(n)*factorial(m-n));
}
\end{chapel}
\end{chapelsource}


